\ssection{Франкендетектив} % задел под детективчик
\subtitle{от команды мёртвой сосны}
\begin{epigraph}
    Mama Weer All Crazee Now!
    \flushright{\normalfont Slade, 1972}
\end{epigraph}
\begin{epigraph}
    Мама, мы все сошли с ума!
    \flushright{\normalfont Виктор Робертович Цой, 1988}
\end{epigraph}

\section*{Склеил ласты}
\begin{epigraph}
Не надо было, нет, не стоило\ldots
\flushright{\normalfont Иван Алексеев, 2013 г.}
\end{epigraph}

\subsection*{Глава 1. Труп на пляже}
--- Итак, что мы имеем?

--- Труп. Белый мужчина. Возраст --- 25-30 лет. Умер в полночь. Не женат, --- ответил эксперт не поворачиваясь.

--- Какое странное положение тела --- как будто его кинули через перила балкона третьего этажа. А после падения он упал на кенгуру и проскакал на нем еще десяток метров, --- пошутил майор, звучно высморкавшись в платок с кенгуру.

--- Меня смущают его фиолетовые волосы. Почему фиолетовый? --- спросил недоумевая старший лейтенант.

--- Наверное из-за автомобиля\ldots Вань, видишь горбатый запорожец ярко красного цвета? Возможно он и стал причиной убийства, ведь сидения в салоне ярко розового цвета, да и отсюда слышно, что в ней до сих пор играет кассета Укупника. Суммируя все вышесказанное, я могу предположить\ldots

--- Ну, товарищ майор, --- взвыл стажёр. --- Извините, что мы приехали сюда не на патрульной машине, а на моей <<ласточке>>, ну а что с трупом то?

--- Сопляк! --- заносчиво начал в ответ старлей. --- Это ж Шерлок Холмс --- вон трубка валяется\ldots

--- Я же вам говорю --- это трубка от водолазного костюма, а не от этого вашего Конан Дойля. И хватит страдать ерундой, давайте лучше улики собирайте.

--- О, а вот ещё маска, --- радостно воскликнул стажёр.

--- Кто-нибудь, объясните мне, почему он в одном ласте? И сделайте уже анализ почвы рядом с трупом, на всякий случай, --- сказал майор экспертам.

--- Факир Рамзанович, мы стоим на пляже, принадлежащем дайвинг-клубу, здесь из почвы только песок и ракушки. Приглядитесь лучше к телу --- у него в руке бумажка с изображением Че Гевары и какой-то непонятной надписью: <<Любимому Семёну на память о былых днях. Крым, 1991 г.>>

--- Что бы это могло значить? Друг? Любовник? И вообще мне его физиономия знакома, где-то я его видел\ldots Вань, а фамилия у трупа-то есть, документы при нём были?

--- Судя по обложке --- эта квашня когда-то была паспортом, но сейчас текст уже не разобрать. На ласте виднеется нацарапанная надпись <<Подоприворота>>. Как думаете, товарищ майор, это его фамилия?

--- Пробьём по базе. Ладно. Коля, пособирай окурки вокруг и заканчиваем.

Светает. Дует холодный ветер в сторону моря. Осень. Кто-то из отошедших в сторонку экспертов настраивает радио и вдруг ловит хороший сигнал. По радио начинает играть свежий хит группы ДДТ <<Что такое осень>>. В салоне машины появляется дружелюбная обстановка.

--- Коля, передай экспертам, чтобы закруглялись, --- сказал Рамзанович спустя некоторое время стажеру. --- Надоело тут торчать, информации особо не прибавляется, да и в участок пора уже.

Стажёр быстрой походкой прошёлся от машины до трупа, передал пожелания экспертам и, попрощавшись с ними, вернулся в машину. Колин запорожец, заскрежетав, двинулся в сторону райотдела.

--- Что-то с этим трупом не так, --- размышлял вслух Ваня. --- И дело даже не в ласте, и не в фиолетовых волосах. Кажется, это не просто несчастный случай. Вероятно, дело будет не таким простым, как выглядит на первый взгляд.

\asterisks

Тот же день. Вечер. Чай. Прокуренный кабинет. Допрос первого подозреваемого.

--- Ты его убил? --- светя лампой в лицо вёл допрос старлей.

--- Я? Убил? Нееет. Я крови боюсь, --- с издёвкой ответил парень, вальяжно откинувшись на спинку стула.

--- Ладно оформляем его\ldots --- бросил Ваня в сторону стажёра.

--- Это что за дела? --- спросил майор, входя в комнату. --- Не, Вань, я всё понимаю, но вот так сразу вешать этого глухаря на пацана, которому завтра стукнет 18, живущего за 30 км от места убийства, а попутно являющегося моим племянником --- это верх некомпетентности. Еще один такой промах и ты отправишься в командировку в Сызрань! Ладно, есть хоть какая-нибудь информация от наших экспертов?

--- Мужчина болел волчанкой и слушал Мираж.

Майор побледнел.

--- Такие совпадения бывают нечасто. В прошлый раз, потерпевший, слушавший Мираж, захлебнулся в ванной. То, что это было самоубийством, выяснилось только через полгода --- свидетелей не было. Оказалось, что мозг умер при прослушивании музыки и парень захлебнулся, впав в кому. Вань, передай экспертам, чтобы проверили наличие воды или других посторонних веществ в легких. Пробили по базе ту украинскую фамилию?

--- После того как стажёр сходил в архив, архива не стало.

--- А нечего курить в архиве! Идиоты! Ладно, значит так. В Сызраньском архиве точно всё есть. Так что, Ваня, у тебя командировочка намечается. И напиши заявление на стажёра, работать же мешает.

--- Пожалейте, Факир Рамзанович, --- взмолился Коля. --- У меня дома кошка, она не выдержит очередной смены работы за этот месяц. Она до сих пор корм вместе с пакетиками ест и вместо воды морозильник облизывает.

--- Очень мило, что ты беспокоишься о своей кошке, но хватит ныть, ядрёна вошь!

Коля молча развернулся и вышел в расстроенном настроении.

--- Корм с пакетиками она ест. Я вообще конфеты не развёртывая ем и ничего. Жив, здоров, спортсмен. Ладно, пусть живёт и работает, может исправится. Мы все через это проходили.

Майор ушёл в себя и начал вспоминать свою молодость.

\asterisks

Всю ночь Ваня провёл в поисках информации о трупе. Прошло уже 24 часа, а дело так и не сдвинулось с мёртвой точки. На следующее утро он зашёл к майору и обнаружил его в крайне неприветливом настроении.

--- У меня для вас 2 новости: хорошая и плохая. С какой начать?

--- Ваня, сейчас 7 утра, вторник. Не думаю, что твоя плохая новость сможет сделать это \emph{замечательное} утро ещё хуже.

--- Ну-с, начнем тогда с плохой. Погоняло убитого: Убитый. Это Рязанский авторитет. Контролировал все проводимые сделки: от передачи партии амфетамина и марихуаны до подпольной продажи консервированных бычков в томате. После продолжительного преследования следственными органами сымитировал свою смерть и следил за всем из подполья, с этим и связано его прозвище. А теперь хорошая новость: в прошлом году его таки поймали и посадили отбывать пожизненное за единственное, в чем его смогли обвинить --- распространение пиратской музыки. Две недели назад он сбежал из колонии и осел у нас тут. Но видимо его нашли ребята из местной звукозаписывающей студии и порешили. Так что у нас есть подозреваемые с мотивом!

\asterisks

На следующий день, обыск звукозаписывающей студии, внезапно оказавшейся знакомым дайвинг-клубом.

Лесоповал. Шатунов. Укупник.

--- Это точно они. Безбашенные ублюдки. Но по внешнему виду комнаты не ясно: то ли они быстро собрались и свалили, то ли этот хаос задаёт атмосферу всей комнате\ldots

--- А мне нравится, --- позволил себе лишнего Николай.

--- А вас, товарищ стажёр, я попрошу не выражаться, --- пресёк панибратское настроение стажёра Иван.

--- Ну всё же по фен-шую, --- не унимался Коля.

--- В Сызрань же отправлю на тренинги парковщиков! --- рявкнул майор и стажёр обиженно удалился из комнаты, на ходу вытаскивая из кармана помятую пачку сигарет.

Ваня оглядел комнату по кругу и его взгляд зацепился за ласты, лежащие под ворохом виниловых пластинок Сюткина. Иван аккуратно достал их и примерил. <<А размерчик-то как раз. Может себе забрать?>>

Внимательно рассмотрев все ласты, Ваня подметил одну особенность --- на каждой левой ласте выцарапана польская фамилия, а на каждой правой --- украинская. Разбив ласты попарно в соответствии с первыми буквами фамилий, он нашел ласту, к которой не хватало пары. На ней было нацарапано <<Пясечны>>.

\asterisks

--- Итак, что мы имеем? Убийство. Подозреваемые: один или группа из нескольких музыкальных пиратов, ведущие свою деятельность под видом дайвинг-клуба. Полные отморозки, предсказать поведение которых совершенно не удаётся. Вряд ли они уже вернутся сюда\ldots --- Ваню прервал резво ворвавшийся в комнату стажёр.

--- Ещё одного нашли!

Майор вместе с Ваней понурив головы спустились вниз и увидели у входа бомжа в ластах, распластавшегося с бутылкой, сжатой в руке. На ластах значилось <<Сидоров>> и <<Smith>>. <<Не местный>>, --- пронеслось в голове у Вани. Он пнул бомжа, на что тот никак не отреагировал. Факир Рамзанович, смеясь над происходящим, посоветовал Коле забрать бутылку. Как только стажёр ухватился за неё, тело резко дернуло бутылку на себя и свернулось калачиком. <<Эх, стажёр, стажёр\ldots>> --- сказал Ваня, и они вернулись в студию.

\subsection*{Глава 2. Засада за садом}
\begin{center}
Серыми тучами небо затянуто,\\
Нервы гитарной струною натянуты,\\
Дождь барабанит с утра и до вечера,\\
Время застывшее кажется вечностью.

~\\
Мы в засаде сидим с умилением,\\
Даги, ингуши --- мы ждём подкрепления.\\
Мы голодаем, но мы выживаем,\\
И целыми днями сидим и страдаем\ldots
\end{center}

--- \ldots какой-то фигнёй\ldots, --- пискляво пропел водила <<буханки>>. 

--- Наверно, всё это из-за чёртовой погоды, --- тихо проговорил стажёр.

--- Да плевать мне на погоду! Хреново на душе становится, когда тебя заставляют несколько дней следить за чёртовым садом. И что в нём такого особенного! Кроме этого чёртового дайвинг-клуба вокруг ни одного здания, ни одной живой души! --- в сердцах ответил водитель Антон.

--- Да ты блин последние сутки только и делаешь, что сидишь со своей книженцией в углу. Небось Жванецкого читаешь, да ржёшь как конь.

--- Неее. Жванецкий --- это прошлый век. Теперь я пишу рассказы в <<Мурзилку>>.

--- Всё равно нихрена не делаешь, --- ответил раздраженно стажёр.

--- Смотри! Кто-то вышел! --- громким шёпотом прервал его Антон.

--- Идиот! Это же их секретарша!

--- А почему у неё борода и кинжал? --- резонно заметил Антон. Он славился своим острым зрением, в отделе его прозвали Соколиным глазом.

--- Она из этих\ldots Блин, забыл как их зовут!

--- Даги что ли?

--- Наверное\ldots Да и какая, блин, разница! Давай смени меня, а я посплю часок другой, --- сказал стажёр и исчез за сидениями.

Дождь на улице только начинал отбивать ритм по крыше уазика, но на душе уже было хреново. Не помогала даже затрёпанная кассета с одним хитом <<Eye of the tiger>>.

Стажёр уже был готов лечь спать вместе со своей любимой плюшевой игрушкой, но тут произошло, то, чего они так долго ждали\ldots Никита наконец-то вернулся с пирожками.

--- Тебя только за смертью посылать! --- недовольно выразился водила.

Опер был весь мокрый и почему-то задыхался. <<Наверное всё-таки бегать за пирожками на другой конец города была плохой идеей>>, --- подумал Никита.

--- Ну я\ldots это\ldots хорошо заблудился\ldots Здесь хрен пойми как подъехать к вашему саду. Дороги петляют, но не заходят во дворы\ldots А всех\ldots всех кого не спроси\ldots

Стажёр вынырнул из глубины салона в надежде присвоить себе пакет с пирожками. Но тут лицо Никиты сильно исказилось в ужасной гримасе, и он смачно чихнул на Антона. Тот успел уклониться, и соплями забрызгало только Колю, который так и остался сидеть, протянув руки за пирожками.

--- Ты, блин, осторожнее будь, --- нервно проговорил Антон. 

--- Т-так и и-инфаркт м-можно по-по-получить, --- оттирал себя от соплей напуганный стажёр. 

--- Хорошо, --- шмыгнув носом и смачно харкнув согласился Никита, открыл дверь и залез в машину.

Дождь всё ещё барабанил как сумасшедший. Крыша протекала, но герои не сдавались. Антон спал за рулём, накрывшись плащом, Никита отогревался после пробежки под дождём, а стажёр\ldots Стажёр принял позу эмбриона, прижав к себе пирожки и плюшевую игрушку.

\asterisks

Стажёр доедал последний пирожок. Как всегда он смотрел фотографии своей кошки и тихонько умилялся. Антон угрюмо следил за входом и только Никита что-то искал в своём чемодане.

--- Та хде же мой блаток, --- гнусавил Никитос. --- Де видели, с кенгуру тагой?

--- Тихо! Началось! --- шикнул Антон.

--- Я же уже говорил, --- раздраженно пробормотал стажёр. --- Это всё ещё персонал.

--- Тоже даги?

--- Да хрен игх поймёшь, --- сказал опер. --- Даги, ингуши\ldots Хто здает.

--- И зачем мы вообще следим за этим садом? --- возмущённо спросил Антон.

--- Ну как же! --- воскликнул стажёр. --- А про труп на пляже слышал?

Следующие два часа прошли в полной тишине и только тихие звуки умиления стажёра периодически прерывали убийственную тишину. 

\asterisks

--- Надо брать! --- громким шёпотом сказал Антон.

--- Колян, проснись! Похоже мы таки дождались! --- сказал Никита, тряся спящего стажёра за плечи. --- Парень идеально подходит под описание: средний рост, чёрная куртка, чёрная шапка, грязь на левом ботинке. Надо брать!

--- Ну та-а-ак, и чего мы ждём? --- зевая спросил Коля. --- Вперёд!

\subsection*{Глава 3. Разбор полётов}
\begin{center}
\note{РОВД №1 г. Владивостока\\
Майору милиции Режневу-Сутуляну Факиру Саиду Рамзановичу\\~\\
РАПОРТ}
\end{center}
\note{В ночь с субботы 31 октября на воскресенье 1 ноября 1992 года было совершено убийство гражданина Российской Федерации Сурина-Семиндяева С. Р. 1953 г. р. В ходе оперативно-розыскных работ была установлена личность подозреваемого. Им оказался гражданин Республики Исландия Ерикссон С., проживающий на территории Российской Федерации с 1980 года. Предположительно, убийство было совершено на национальной почве, орудием убийства послужил отравленный клей, следы которого были обнаружены на ласте убитого.

В ходе следственного эксперимента была восстановлена картина происшествия. 31 числа около 6 часов вечера убитый посетил дайвинг клуб <<У затонувшего дайвера>> с целью прохождения курса повышения квалификации. Убийца предварительно залил отравленный клей в левый ласт потерпевшего. К моменту прибытия убитого клей успел высохнуть и не вызвал у него никаких подозрений. Оказавшись в воде, потерпевший почувствовал легкий дискомфорт, вызванный химической реакцией клея с водой. Перед смертью убитый смог самостоятельно выбраться на пляж, и предпринял попытку снять ласты. По всей видимости, правый ласт снять удалось, а левый к тому моменту намертво приклеился к его ноге. После этого убитый потерял сознание и скоропостижно скончался.

Также была установлена личность предполагаемого заказчика убийства. Им оказался гражданин Республики Грузия Соколов Ш. А., являющийся владельцем упомянутого выше дайвинг клуба.

~\\
\noindent Старший лейтенант\\
Татаренко И. П. \hfill {7 ноября 1992 г.}}

\subsection*{Глава 4. В объятиях Морфея}
Ваня дописал рапорт и отложил его в сторону. На сегодня вся работа была сделана. Откинувшись на спинку стула, он предался воспоминаниям о былом.

Это были 70-е. Мы выживали как могли. Как-то один мой товарищ по детдому надыбал где-то жвачки забугорной. И делиться ни с кем не хотел, но я был умён не по годам и нашёл-таки способ уломать его на четверть пластинки. После отбоя, лёжа в кровати, разжевал я её хорошенько. Вкус у неё был отвратный, как у гуталина, смешанного с солидолом. Но я не мог себе признаться в этом и продолжал отчаянно её жевать, до тех пор, пока её вкус совсем не исчез. И тут я понял, что жую кнут! Огляделся вокруг --- я на лошади, стою на краю каньона, а на той стороне краснокожие намекают мне уйти отсюда, размахивая томагавками и крича что-то непонятное. Дёргаю лошадь за поводья, пускаю её в галоп по прерии. Скачу прямо, кручу головой по сторонам в надежде найти какой-нибудь городишко. Безуспешно.

Вскоре лошади поплохело --- из её рта начала вырываться пена, ритм её шага сбился. Через минуту она упала замертво. Я оказался в абсолютно безнадежной ситуации. Выбрав случайное направление, я поплёлся, позвякивая шпорами на своих сапогах. Спустя час, а может два, я увидел одиноко стоящую таверну. Воодушевленно я зашел внутрь.

То, что я увидел, не поддавалось никакому объяснению. Внутри было уходящее в бесконечность ледяное озеро. По поверхности льда катались императорские пингвины, напевающие песни Beatles. Под водой виднелись замерзшие гигантские черви, чьи морды были ориентированы параллельно плоскости льда. Сделав пару шагов внутрь <<таверны>>, я почувствовал резкую боль в левой пятке. Подняв ногу, я увидел блок lego вместо каблука. В этот же момент, от моей правой ноги начала разветвляться сеть трещин на поверхности застывшей воды. Буквально через секунду я оказался подо льдом с червями и всеми вытекающими\ldots Переохлаждение не заставило себя долго ждать: краски мира начали меркнуть, я начал терять ориентацию в пространственно-временном континууме.

Очухался я в кровати детдома, весь в холодном поту. На меня смотрели все окружающие собратья по несчастью и какой-то мужик в красно-черном свитере, сидящий у меня на кровати. Позвякивая лезвиями перчатки на правой руке, он загадочно расплылся в улыбке: <<Пришло время поиграть>>. Решив для себя, что играть мы будем в догонялки, я бросился наутек, заметно похрамывая на левую ногу. Нервно оглядываясь назад, я увидел, что мужик неторопливо встал и пошёл вслед за мной.

На моей стороне было то, что я умело ориентировался в лабиринте коридоров нашего корпуса. Я спрятался в чулане за очередным поворотом. Из-за двери доносились высокие голоса девочек, напевающие странную песню, более похожую на считалочку: <<Раз, два, Фёдор заберёт тебя; три, четыре, запирайте дверь в квартире, пять, шесть, Фёдор хочет всех вас съесть, семь, восемь, кто-то к вам придёт без спросу, девять, десять, никогда не спите, дети\ldots>> На последнем слове дверь чулана разлетелась в щепки и передо мной предстало три зловещих силуэта: здоровенный кот, доберман, и тот самый приземистый мужичок в полосатом свитере.

--- Да что ты бегаешь от нас? --- спросил мужик.

--- Ты нам посылку просто отдай и всё --- продолжил кот. --- Вот мы тебе из поликлиники документы принесли, а дядя Фёдор в паспортный стол сбегал.

Внезапно я обнаружил у себя под ногами коробку с фоторужьём, а доберман пролаял: <<Это мне прислали>> и протянул какую-то справку. Не зная, что делать в подобной ситуации, я не раздумывая отдал посылку и отправился было в постель. Но гости не спешили меня отпускать и предложили меня щёлкнуть на память. Не испытывая к ним особого доверия, я отказался. Тогда они настоятельно предложили мне попить с ними чаю.

До столовой мы не дошли --- на нас бежали знакомые мне по каньону краснокожие, по-прежнему размахивая томагавками. Долго бежать нам не удалось: сначала они схватили кота, затем Фёдора, а затем и пса, бросившегося на выручку мужику. Мне же снова удалось скрыться в том же самом чулане. От страха я закрыл глаза. Как только я перестал слышать грозные голоса индейцев, я открыл глаза, облетел найденную мной поляну, усеянной цветами всех видов и размеров. Взобравшись на ромашку, я измазался в пыльце. Отобедав нектаром, я увидел своё отражение в капле росы --- я самый красивый шмель на этой поляне! Меня тут же окружили бабочки, но по их ухмылкам я понял --- что-то здесь неладно! Они яростно и методично теснили меня с этой поляны, давая понять, что мне здесь не рады. Так я оказался на опушке леса, одинокий, всеми забытый, без шансов выжить. На небе сгущались тучи, раскаты грома наводили ужас, а порывы ветра не давали мне спрятаться в листве. И тут я почувствовал холодок, пробежавший между моих крылышек. Это осоед пристроился мне в хвост. На его стороне была скорость, на моей --- маневренность, фора в 2 метра и порывы ветра. Очередным порывом меня ударило о землю и я отключился.

Очнувшись, я посмотрел наверх и увидел над собой склонившегося майора в чёрном балахоне. <<Ты знаешь, что случилось с твоим отцом?>> --- спросил он лежащего мальчика. <<Ты убил его!>> --- крикнул я. <<Нет, я --- твой отец!>> --- ответил майор, протягивая руку --- <<Присоединяйся ко мне, и мы вместе будем править отделом как отец и сын\ldots Это твоя судьба.>> Поднявшись, я попытался побежать, но ослепленный солнечным бликом от блестящей поверхности звезды в погоне майора оступился, упал, и непередаваемое ощущение от удара качелькой по лбу заставило меня проснуться.

<<Не может быть\ldots>> --- воскликнул Ваня, поднимая голову со стола.

\subsection*{Глава 5. Во все тяжкие}
В городе два ведущих дайвинг-клуба полностью разделили рынок между собой. И только РОВД, последний оплот свободы и порядка в этих краях, обеспечивал спокойствие в городе. Но убийство на пляже сместило равновесие сил: англо-руссы предъявили претензии укро-полякам на почве потери своего бойца. Всё оказалось не так просто --- это была подстава, тщательно спланированная племянником майора милиции, который также случайно оказался наследником англо-русского дайвинг-клуба.

После того, как племянника майора отпустили из отделения, на него вышли укро-поляки и нанесли ему несколько огнестрельных ранений. В последние минуты своей жизни он сказал врачам, что забыл выключить утюг и принял анонимный звонок, в котором ему сообщили, что он только что стал директором англо-русского дайвинг-клуба.

Майор, узнав о случившемся, начал мстить, используя свои обширные связи и связи своего пропавшего брата\ldots

Ваня проводил расследование, до самого конца не представляя, какую роль во всём этом играет майор. В итоге, майора посадили, а Ваньку и стажёра повысили в должности за раскрытие этого дела.

\subsection*{Глава 6. Бесконечная осень}
Майор проснулся от звонка. Ванин голос в трубке устало промямлил: <<Факир Рамзанович, на пляже нашли труп. Мы с Николаем сейчас за вами заедем.>> Майор устало поднялся и начал собираться. <<В такую погоду только ехать на пляж, расследовать убийство>> --- пронеслось в голове Факира Рамзановича. Голова гудела от недосыпа и насморка. Горбатый запорожец Коли вкатился во двор через пару минут. Недовольный Факир загрузился на заднее сидение, и спустя 15 они были на месте. Подходя к копошащимся судмедэкспертам, майор спросил:

--- Итак, что мы имеем?

\subsection**{Настоящий рок-н-ролльщик}
--- Вот так всё и было --- заплетающимся голосом закончил свой рассказ незнакомец. Честно сказать, я не верил ни одному его слову. Он попытался встать, но бремя выпитого ерша не позволил ему этого сделать. Внезапно для самого себя он грузно осел на стул, и из его кармана выпали документы. Я поднял их и увидел среди них удостоверение капитана полиции Демьянова Николая Руслановича и пропуск в дайвинг клуб на имя Ступницкого Владимира Дмитриевича\ldots
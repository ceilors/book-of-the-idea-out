\section{Общество временно живых писателей}

\asterisks
Я сидел в тёмной и сырой комнате. По грязно-синей стене стекали капли конденсата. Меня до костей пробирал жуткий холодок, доносившийся из-за ржавой металлической двери. В середине этого всего стоял странный стол, как будто, тот кто его собирал, перепутал икеевский чертеж, но всё-таки смог закончить эту странную поделку. На нем стояла, такая же неординарная по форме лампа...

Тут внезапно зашёл человек в форме... прокашлялся и сел на стул, который стоял позади этого стола. Я не смог выдержать этой гнетущей атмосферы и резко выразился...

-- Вы ничего не докажете!
-- Так P всё-таки равно NP?

\asterisks
Я очнулся... или нет? Что происходит? Где это я? Эти слова как будто застряли у меня в голове. Понемногу ко мне начало возвращаться сознание... Сидел я в крохотной комнатушке. На полу лежал полузатёртый кафель и свет судорожно помаргивал, будто сумасшедший настукивает музыкальный ритм... Может быть он что-то хочет мне сказать? Я не знал наверняка...

-- Я НЕ ЗНАЮ АЗБУКУ МОРЗЕ, -- прокричал неожиданный голос...
Я резко обернулся, но никого не было. Только потрёпанная ванная и пара пакетов с мусором... Мне почему-то стало очень страшно...

-- Кто это сказал?, — дрожащим невнятным голосом пробормотал я.

Но ответа не было.... И тут до меня начало доходить -- это был мой голос...

Сквозь сон я вскочил с кровати. С меня лил холодный пот... в комнате царил тихий полумрак. Я посмотрел на часы, на которых было 06:32. Окинув взором всю комнату я только заметил своего любимого кота Тишку, который потягиваясь шел ко мне. Он наверное рассчитывал, что я его сейчас буду его кормить. В комнате кроме меня и Тишки никого не было. Да и откуда бы завелось бы. Я всё основное время был занят своей работой, а вне рабочего времени хикковал дома.

Я встал с кровати и пошёл к холодильнику.
-- Ну и приснится же всякая хрень... даже жрать захотелось, -- сказал я подходя к холодильнику.

-- Ага, со всеми бывает, -- сказал я сидя на стуле.

-- Я умер? -- спросил у него-себя.

-- Расслабься: это парадокс Банаха-Тарского.

\asterisks
Смеркалось. Лес окружал меня. Лес был кругом. Казалось на планете больше ничего не осталось кроме этого леса. Мохнатые ели нещадно стегали меня по лицу, когда я продирался сквозь них в надежде найти тропу. Но я всё больше забредал куда-то вглубь. Я потерялся. Окончательно. Мелкий дождь жалил меня холодными каплями свинцово-серой воды. В ботинках давно хлюпало. Осень в лесу прекрасна только на картинах художников. Реальность жёстче, как эта колючая хвоя.

Ведро с грибами было давно брошено. Нож я потерял после 3-го часа блуждания. Я устал; меня клонило в сон.

Передвигаясь от дерева к дереву, я обходил каждое, пытаясь сориентироваться, а затем шёл к следующему. И когда мне начало казаться, что это никогда не закочится...

... я вышел из леса.

\asterisks
Мы решили поймать 57 маршрутку. Мы -- это я и Илюша, мой бородатый друг. У каждого дрыща без растительности на лице должен быть низкий бородатый друг. \( 5+7=12 \), а \( 5*7=35 \). Это взаимно простые числа, поэтому я подумал, что поймать именно её будет удачной идеей. Но я никогда ещё так не ошибался. Илюша высказал идею сесть на конечной, благо она была недалеко -- метров 800 наверное. И мы пошли. В темноте. Через лужи и грязь. Это второй Волгоград, вас не готовили к этому нигде. Не дойдя до конечной метров 150 мы увидели отъезжающую 57. ``Не последняя же'' -- подумали мы и направились дальше. Мы ещё никогда так не ошибались. Дойдя до конечной мы прождали маршрутку полчаса. Это были самые невыносимые полчаса в моей жизни. Я оказался один на один со страхом и ненавистью на втором Волгограде. Через полчаса у Илюши сдали нервы и он согласился с моей идеей вернуться. Так мы очутились на перекрёстке Огарёва и Социалистической. Через 50 минут после зарождения идеи о 57 маршрутке. Здесь мы уже ждали 84-ую. \( 8 + 4 = 12 \), \( 8 * 4 = 32 \). \( 32 - 12 = 20 \), именно столько мы ждали её. Но было уже 8. А это значит, что проезд 30 рублей независимо от расстояния. Тупой платит в 1,5 раза больше.

\asterisks
Жёлтые стены. Жёлтые стены и потолок с обсыпающейся штукатуркой. 4 дня. Это первое. Это последнее, что я вижу перед. После сна. За поцарапанным и заплёванным самим временем стеклом окна палаты виднеется почерневший сад. Мёртвый. Осень.

Тускло. Лампочка на 40 Вт не спасает. Поглощает тоска. Тьма сжирает моё естество. Душу? Нет. Я материалист же. Вроде. Образован. Где моя медаль?! Филдс! В понедельник ел вареники и бац! Звонок и я медалист! Мендалист. Мендаль...

[Вошли врач и медсестра]

-- Ну как вы, голубчик? -- улыбается -- Зиночка, речь по-прежнему несвязна?

-- Он всё что-то бормочет про контуры и замыкание...

-- Ясно. 5 кубиков... электрику

\asterisks
Из конца комнаты только и доносилось протяжные и хриплые слова:

-- Каждые два.
Мне казалось что я схожу с ума... Темп с каждым произнесенным словом будто ускорялся...

-- Каждые два... Каждые два…
И тут он встал и пошел на меня. У меня все сжалось от предстоящего ужаса, который он мог вывалить на меня. Я вжался в кресло и ожидал наихудшего исхода...

-- Андрей Васильевич, -- проговорил он жалобно вполголоса.

-- Все, мне конец, -- подумал я.

-- Ну не могу я выучить ряд Фибоначчи, не понимаю я эту вашу галиматью.

-- Ничего не поделаешь, -- с раздражением сказал я, -- через неделю на пересдачу!

\asterisks
Это случилось поздней зимой... По моему мнению я тогда еще мало разбирался в этом, но все произошло слишком быстро. Мои сокурсники презрительными взглядами провожали меня, как будто хотели сказать что-то плохое, но из-за сложившейся ситуации не могли.

-- Это опять повторилось, -- сказал я про себя, -- не в 4-ый раз же!
Когда я уже вышел из поля их зрения, то меня догнал мой лучший друг Василий.

-- Тоже смог?, -- спросил я у него.

-- Ага, но как у тебя точно не смогу! Это все же автомат по матану!

\asterisks
-- Давайте начнём с самого начала, -- сказал судья. -- Что же там на самом деле произошло.

\begin{center}\( \#\#\# \)\end{center}

Это был довольно свежий день для того ``адски'' жаркого лета. Птички щебетали направо и налево, солнышко грело, но не обжигало.
Всё было слишком хорошо...

\begin{center}\( \#\#\# \)\end{center}

-- Пожалуйста приступайте к основной части... Мы же не на литературном вечере собрались.
-- Да, конечно.

\begin{center}\( \#\#\# \)\end{center}

Я только что вышел из университета... усталый... Всё-таки пять пар это не хухры-мухры.
И тут ко мне подошёл потрепанный жизнью человек в довольно простом костюме...
-- Молодой человек... Не хотите книжечек по квант.меху взять? Совсем новые, отдам за пол цены...
Мне они не пригодились, может вам нужны будут!
-- За сколько отдашь?, -- спросил я.
-- За 500 все три можешь забрать.

В то время я ещё не подозревал какая меня ждёт беда... Я радостный расплатился за книги, не каждый день можно получит такие хорошие книги, да и по такой низкой цене. Я даже не задумался открыть их и посмотреть...

Вернувшись домой я положил пакет с книгами на тумбочку и пошёл готовить себе ужин.

Часов около 20 я решил насладится чтивом одной из книг. Взяв первую попавшуюся из пакеты и открыв её я ужаснулся...

Книга была исписана замечаниями и правками... Почти каждый абзац был откорректирован...

Это был второй тревожный звонок...

-- Хорошо, что это всего лишь карандаш, -- подумал я.

Но что-то мне не давало покоя... Я не стал стирать записи предыдущего владельца, а просто игнорировал их.
Но... но, у меня не получилось...

Они доводили меня, насмехались... будто говорили какой я идиот... Во мне появилось какое-то незнакомое чувство...
Раздражение, ярость...

Через 2 часа этих мук мне всё ещё хотелось чем-то возразить... Но это же всего лишь надписи в книжке...

Я решил лечь спать...

На следующее утро я встретил того же мужчину. Я было хотел вернуть свои деньги, но что-то изменилось после того, как он заговорил со мной?

-- Ну что? Как оно?, -- спросил он ехидно.
-- Я в ярости... всё неправильно..., -- я пытался говорить складно и внятно, но у меня не получалось.

Тут заметив двух полицейский он как ужаленный быстро скрылся в ближайшем переулке, только остались от него пара тёмных пакетов с книгами.

-- Нарушаем закон гражданин?, -- сказал один из офицеров. -- Нелегальные деяниями занимаемся?
-- Я.. я.. НЕТ!! Я честный... честный гражданин.

\begin{center}\( \#\#\# \)\end{center}

-- Думаю дальше вы знаете как обстоят дела господин судья...

\begin{center}\( \#\#\# \)\end{center}

Вы слушали радио-пьесу: ``Мир, где обсуждения запрещены законом.''
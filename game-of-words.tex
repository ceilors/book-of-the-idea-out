\section{Рубрика n-смысленности + The game of words}

\begin{displayquote}
    \begin{flushright}
        \emph{--- Пойдём до комсы или до чекистов?\\
        --- Просто пошли, а там как пойдёт.}\\
        Из разговора двух прохожих, 3 сентября 2016
    \end{flushright}
\end{displayquote}

Классика двухсмысленности:
\begin{itemize}
    \item гонять чаи
    \item заварит кашу
    \item бросаться в глаза
    \item убивать время
    \item бисер метать
    \item волынку тянуть
    \item время истекло
    \item долгий ящик
    \item зарубить на носу
    \item ...
\end{itemize}

\begin{flushright}
    [И всё-таки \emph{фразеологизмы} вещь хорошая!]
\end{flushright}

\begin{flushleft}\parskip1em
    Правила отбора от Бора.

    Парень с Курил скурил все сигареты в блоке, сидя сутками с утками в блоке общежития, и из-за этого теперь почти в агонии ехал в вагоне.

    Он думал полететь в Тулузу, закатывая последний шар партии в ту лузу.

    Born, born in 1970, was a cool men.

    Смог смог помешать движению в городе. (\emph{ну или просто}) Смог который смог.

    You may shelter in our office off ice time (вы можете согреться в здании нашего офиса в холодные часы)

    Mess-age (испорченный возраст) it's time when a message (соц.сети) it's main in the life of teenagers.

    Вопрос: \emph{что означает Б. в имени Бенуа Б. Мандельброт?}\\
    Ответ: \emph{Бенуа Б. Мандельброт.}

    Подрубрика "Помощь молодому писателю" (начало какой-нибудь повести): набор заготовок для романов/дедективов/триллеров/...

    Распродажа уцененных персонажей 1-го и 2-го плана, а также 80\% скидка на залежавшиеся финалы для комедий.

    Германия. Герман и я, выйдя из аэропорта в это хмурое утро, оказались перед забором, который, в свою очередь, располагался за бором...

    \emph{4-смысленная фраза:} Хватит мять булки!

    Де Бройля всю жизнь волновали элементарные частицы.

    Штирлиц подсыпал яд врагу в рагу.

    Рентген любил просвещать людей.

    КОТЭ --- классно обманул товарища экзаменатора.

    \emph{--- Ошибка, которая привела к проигрышу партии!\\
    --- Да, в 91-м году...}
    \vspace*{-1em}\begin{flushright}
        Из разговоров за бильярдным столом, 11 сентября 2016
    \end{flushright}

    Та волга потерялась между полем, где росла таволга, и крутым берегом Волги. % так себе, но пойдёт

    \emph{Мама, я больше не Будда!}
\end{flushleft}

\subsection{Recursive acronym}
\begin{flushleft}\parskip1em
    Рекурсивный акроним --- бэкроним (аббревиатура или акроним), который косвенно или напрямую ссылается на себя.\\
    Классика:\\
    ЛОМ --- лом обыкновенный металлический.\\
    GNU --- GNU's Not UNIX.

    \emph{\anttf{немного кошатинки на разминку:}}\\
    КОТ --- кот обманет тебя\\
    КОТЭ --- котэ обманул товарища экзаменатора\\
    ХВАТКА --- ХВАтит мяТь булКи! А!\\
    КРУТО --- круто разработал усовершенствование текущего опуса

    \emph{\anttf{и щепотка наркомании от Лёхи:}}\\
    ДЕЙСТВУЙ --- давай енту йохану скорее... также вувузелу у Йорика.\\
    ТИРЕ --- так и рождаются еноты.

    \emph{так и рождаются диалекты}\\
    STAR --- star to a rise\\
    \emph{если перевести rise как рассвет}
\end{flushleft}

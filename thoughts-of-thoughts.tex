\pagecolor{white}
\section*{Мысли о <<Мыслях>>}
% надо придумать стиль оформления

\begin{flushright}
\emph{Да я просто хотел картинку показать, а оно вон как...}\\
    Алексей.пп % пп --- практикующий поэт/писатель/программист... [тут немного в твоём стиле описание сделал;)]
\end{flushright}
\vspace{1cm}
\begin{flushright}
\emph{Просто хотелось создать нечто для завязки разговора...}\\
Антон.мм % мм --- мёртвый математик [имеется ввиду, что, т.к. для профессиональной математики я уже слишком стар (<<шутка про суперстара>>), то во мне уже умер математик]
\end{flushright}
\vspace{1cm}
\begin{flushright}
\emph{Мыслей не измышляю.}\\
Владимир.фф % фф --- фактический физик
\end{flushright}
\vspace{1.5cm}
\begin{figure}[ht!]
    \Large
    Если бы я был известным писателем, я бы выпустил книгу с десятью различными концовками, которые пошли бы на печать с разной степенью редкости, но не рассказал бы фанатам об этом, чтобы я мог наблюдать их растерянность, то как они расходятся во мнениях о том, как закончилась история. Потом, когда они бы поняли это, я бы признался, сказав им, что я выпустил 11 альтернативных концовок, и смотрел бы, как они снова паникуют, пытаясь найти последнюю концовку.
    \caption{\% данная книга выйдет в 2-х версиях: pdf для обычных гипотетических читателей и tex для (ещё менее) гипотетических фанатов}
\end{figure}
